\documentclass{article}
\usepackage[utf8]{inputenc}

\documentclass[a4paper]{article}
\usepackage[12pt]{extsizes}
\usepackage{amsthm, amssymb, amsmath, amsfonts, nccmath, empheq}
\usepackage{float}
\usepackage[hidelinks]{hyperref} 
\usepackage{color,colortbl} 
\renewcommand{\labelenumii}{\Roman{enumii}}
\usepackage[warn]{mathtext}
\usepackage[T1,T2A]{fontenc}
\usepackage[utf8]{inputenc}
\usepackage[english,russian]{babel}
\usepackage{tocloft}
\linespread{1.5}
\usepackage{indentfirst}
\usepackage{setspace}
\usepackage{minted}
\usepackage{csquotes}
\usepackage{movie15}
%\полуторный интервал
\onehalfspacing


\newtheorem{theorem}{Теорема}
\newtheorem{definition}{Опредление}
\newtheorem{corollary}{Следствие}[theorem]
\newtheorem{lemma}{Лемма}


\newcommand{\RomanNumeralCaps}[1]
    {\MakeUppercase{\romannumeral #1}}

\usepackage{amssymb}
\usepackage{epigraph}
\usepackage{graphicx, float}
\graphicspath{{pictures/}}
\DeclareGraphicsExtensions{.pdf,.png,.jpg}
\usepackage[left=20mm,right=1cm,
    top=2cm,bottom=20mm,bindingoffset=0cm]{geometry}
\renewcommand{\cftsecleader}{\cftdotfill{\cftdotsep}}

\addto\captionsrussian{\renewcommand{\contentsname}{СОДЕРЖАНИЕ}}

\usepackage{fancyhdr}
\usepackage[nottoc]{tocbibind}

\fancypagestyle{plain}{%
\fancyhf{}
\renewcommand{\headrulewidth}{0pt}
\fancyhead[R]{\thepage}
}

\usepackage{blindtext}
\pagestyle{myheadings}
% href
\usepackage{hyperref}
\setcounter{MaxMatrixCols}{20}


\begin{document}
\begin{titlepage}
  \begin{center}
  
     
    \large
    
    Санкт-Петербургский политехнический университет Петра Великого
    
    Институт прикладной математики и механики
    
    \textbf{Высшая школа прикладной математики и вычислительной физики}
    
    \vfill
     
     
    \textsc{\textbf{\Large{Отчёт о летней производственной практике}}}\\[5mm]
     
    {\large \textbf{Практика по получению профессиональных умений и опыта в профессиональной деятельности}}
    
    \\ на тему:\\ <<Изучение методов моделирования эпидемии COVID-19>>\\

\end{center}

\vfill

~~Место выполнения:~~~~~~~~~~НИЛ <<Математическая биология и бионинформатика>>, ИПММ \\

\begin{tabular}{l p{140} l}
Выполнила студентка \\группы 3630102/80401 &  &\underline{\hspace{3cm}}   А. С. Мамаева\\\\

Оценка научного \\ руководителя &  & \underline{\hspace{1.5cm}} \underline{\hspace{2cm}}   В. В. Гурский \\\\

Должность и место работы\\ научного руководителя & & к.ф.-м.н., с.н.с., ФТИ им Иоффе,\\  
&& НИЛ <<Математическая биология и\\ &&бионинформатика>>, ИПММ \\\\
\end{tabular}

\hfill \break
\hfill \break
\begin{center} Санкт-Петербург \\2021 \end{center}
\thispagestyle{empty}
 
\end{titlepage}
\newpage
\begin{center}
    \setcounter{page}{2}
    \tableofcontents  
\end{center}


\newpage

\epigraph{На самом деле, вся эпидемиология, связанная с изменением заболеваемости время от времени или от места к месту, должна рассматриваться математически, независимо от того, сколько переменных в ней задействовано, если ее вообще можно рассматривать с научной точки зрения.}{Сэр Рональд Росс}

\addcontentsline{toc}{section}{Введение}
\section*{Введение}
\noindent В новом тысячелетии человечество столкнулось с инфекционными болезнями, о которых никто не знал. На смену чуме и тифу пришли опасные вирусы. Изменение окружающей среды, потепление климата, увеличение плотности населения и другие факторы провоцируют их появление, а высокая миграционная активность населения способствует распространению по всему миру. Поистине, инфекции не знают границ.\\\\
\noindent Официальная информация о вспышке пневмонии неизвестной этиологии в городе Ухань, столице провинции Хубэй, появилась впервые 31 декабря 2019 г. из центра Всемирной организации здравоохранения (ВОЗ) в Китае. Уже 3 января 2020 г. новое заболевание было подтверждено у 44 пациентов. Все они — взрослые жители города Ухань, связанные с местным рынком животных и морепродуктов Хуанань. 7 января 2020 г. учеными из Шанхайского клинического центра общественного здравоохранения и Школы общественного здравоохранения была установлена полная геномная последовательность возбудителя этой пневмонии — нового штамма коронавируса, получившего временное
название \textbf{2019 Novel coronavirus (2019-nCoV)}, который, по мнению экспертов ВОЗ, не был ранее
идентифицирован. 11 февраля 2020 г. новая коронавирусная инфекция получила название \textbf{COVID-2019} (\textit{COrona VIrus Disease 2019}, коронавирусная болезнь 2019 года), а вызывающий ее вирус был переименован в \textbf{SARS-CoV-2} (\textit{Severe acute respiratory syndrome coronavirus 2}, второй коронавирус тяжелого острого респираторного синдрома).

\addcontentsline{toc}{section}{Этиология и патогенез}
\section*{Этиология и патогенез}
\noindent \textit{Коронавирусная инфекция} — острое вирусное заболевание с преимущественным поражением верхних дыхательных путей, вызываемое РНК — содержащим вирусом рода \textit{Betacoronavirus} семейства \textit{Coronaviridae}. Коронавирусы (лат. \textit{Coronaviridae}) — семейство, включающее на январь 2020 года 40 видов РНК содержащих сложно организованных вирусов, имеющих суперкапсид. Объединены в два подсемейства, которые поражают человека и животных. Название связано со строением вируса: из суперкапсида выдаются большие шиповидные отростки в виде булавы, которые напоминают корону. Вирионы размером 80-220 нм. Нуклеокапсид представляет собой гибкую спираль, состоящую из геномной плюс-нити РНК и большого количества молекул нуклеопротеина N. Имеет самый большой геном среди РНК-геномных вирусов. В его структуре выделяют суперкапсид, в который встроены гликопротеиновые тримерные шипы (пепломер), мембранный гликопротеин, малый оболочечный гликопротеин, гемагглютинин эстеразу.
\begin{figure}[H]
    \center{\includegraphics[scale=0.3]{VIRUS.jpg}}
	\caption{Строение коронавируса}
	\label{fig:image}
\end{figure}

\noindent В настоящее время известно о циркуляции среди населения четырёх коронавирусов (HCoV-229E, -OC43, -NL63, -HKU1), которые круглогодично присутствуют в структуре ОРВИ, и, как правило, вызывают поражение верхних дыхательных путей лёгкой и средней степени тяжести. Наибольшее число случаев коронавирусной инфекции регистрируется в зимнее и весеннее время. Источник этой инфекции — заболевшие и носители. Пути передачи — воздушно-капельный и контактно-бытовой, реализуемый через контаминированные коронавирусом предметы обихода.

\section{Основная часть}
\subsection{Постановка задачи}
\noindent Данные по динамике заболевания COVID-19 в разных странах обновляются в ежедневном режиме и доступны для построения и исследования эпидемиологических моделей. Необходимо ознакомиться с методами построения эпидемиологических моделей SIR и SEIRS и программой по реализации алгоритма SEIRS на языке Python. Необходимо обучить модель на таких регионах России, как: Башкортостан, Татарстан, Краснодарский край, Крым, ХМАО-Югра. А также провести сравнительный анализ характерных параметров эпидемии в разных локациях.

\subsection{Описание моделей}

\noindent Эпидемии издавна угрожали человечеству, и только в ХХ веке были разработаны эффективные средства борьбы с инфекциями. К числу этих средств принадлежат и системы дифференциальных уравнений — математика помогает моделировать распространение эпидемий и помогает понять, как следует с ними бороться.

\subsubsection{Модель SIR}

\noindent Во время, так называемой, <<девственной эпидемии>> у нас изначально есть популяция из $N$ человек, подверженных риску заболевания, которые составляют восприимчивую группу. Если в эту популяцию попадает человек с трансмиссивным патогеном, то со временем некоторые из восприимчивых людей будут инфицированы и станут частью инфекционной группы, члены которой будут способствовать дальнейшей передаче. В зависимости от возбудителя заболевания люди могут выздоравливать и приобретать иммунитет, который может быть пожизненным (например, в случае кори) или временным (например, в случае гриппа).\\

\noindent Модель \textbf{SIR (Susceptible, Infected, Recovered)} является базовой для описания распространения инфекционных заболеваний и была предложена в 1920-х годах шотландскими эпидемиологами Андерсеном Кермаком и Уильямом Маккендриком. Согласно SIR, население делится на три группы: восприимчивые ($S$), инфицированные ($I$) и выздоровевшие ($R$). Размеры этих групп зависят от времени $t$. Во время девственной эпидемии мы часто предполагаем, что распространение происходит настолько быстро, что мы можем игнорировать любые изменения в популяции из-за рождений и смертей - это так называемая <<закрытая эпидемия>> с фиксированным размером популяции $S + I + R = N$. Модель SIR не имеет вероятностного компонента, за исключением предположения, что совокупность может смешиваться случайным образом и достаточно велика, чтобы можно было использовать прогнозы, основанные на средних частотах. \\

\noindent Модель инициализируется со всей популяцией, входящей в восприимчивую группу, за исключением одного инфекционного индивидуума: $I(0) = 1$,$~S(0) = N - 1$, $~R(0) = 0$. В каждую единицу времени (например, день) любой инфицированный человек может контактировать с $k$ другими людьми в популяции. В среднем за единицу времени они будут контактировать с $\frac{kS}{N}$ людьми, восприимчивыми к инфекции, и заражать $\frac{k\pi S}{N}$ из них, если $\pi$ - вероятность заражения при контакте. Обычно $k$ и $\pi$ объединяются в скорость передачи $\beta=\pi k$, которая представляет собой среднюю скорость, с которой инфицированный человек может заразить восприимчивого. Инфицированные люди выздоравливают с постоянной скоростью $\gamma$, а $\frac{1}{\gamma}$ - инфекционный период (или среднее время выздоровления).\\

\noindent Эти скорости изменения каждой группы можно записать в виде трех связанных дифференциальных уравнений:

\begin{center}
    \begin{cases}
      \frac{dS}{dt} = -\frac{\beta S I}{N} \\
       \frac{dI}{dt} = \frac{\beta S I}{N} - \gamma I\\
       \frac{dR}{dt} = \gamma I\\
    \end{cases}
\end{center}
\\

\noindent Первое уравнение системы означает, что изменение числа здоровых (и при этом восприимчивых к заболеванию) индивидуумов уменьшается со временем пропорционально числу контактов с инфицированными. После контакта происходит заражение, восприимчивый переходит в состояние инфицированного. Второе уравнение показывает, что скорость увеличения числа заразившихся растет пропорционально числу контактов здоровых и инфицированных и уменьшается по мере выздоровления последних. Третье уравнение демонстрирует, что число выздоровевших в единицу времени пропорционально числу инфицированных. Иначе говоря, каждый заболевший через некоторое время должен поправиться.\\

\noindent Уравнения обычно решаются численно. На данный момент есть несколько ключевых наблюдений, основанных на их формах. Вспышка начнется, если изначально $\frac{dI}{dt} > 0$, что означает $\frac{N}{S(0)} < \frac{\beta}{\gamma} $ или, если $S(0) \approx N$, то $\frac{\beta}{\gamma}>1$. В этом случае начальное увеличение $I$ будет экспоненциальным со скоростью $r=\log \frac{\frac{\beta}{\gamma}}{G}$ и временем удвоения $\frac{\log 2}{r}$. Здесь $G$ называется последовательным интервалом и представляет собой среднее время между последовательными случаями в цепочке передачи. На многие важные аспекты динамики эпидемии влияет соотношение $\frac{\beta}{\gamma}$, называемое основным репродуктивным числом ($R_0$), которое представляет собой ожидаемое количество вторичных случаев, вызванных одним первично инфицированным индивидуумом, введенным в популяцию без предшествующего иммунитета. Согласно соглашению, мы выражаем $S$, $I$ и $R$ в процентах от общей численности популяции $N$. Ключевой особенностью $I$ является местоположение и размер пика инфекции, который возникает при $S=\frac{1}{R_0}$ и определяется как $I_{max}=1-\frac{1+\log R_0}{R_0}$. Вторая важная характеристика — это общее количество инфицированных - совокупный размер эпидемии $R(\infty)$. Независимо от значения $R_0$, эпидемия самозатухнет $(I(\infty) \rightarrow 0)$, если в популяцию не будут добавлены новые восприимчивые люди (в результате рождения или потери иммунитета). Это происходит потому, что со временем выздоровление начинает опережать инфекцию до того, как будут инфицированы все оставшиеся восприимчивые люди. Таким образом, модель предсказывает, что будет часть людей, $S(\infty)$, которые избегут инфекции, заданной неявным уравнением $S(\infty) = e^{-R_0 (1 - S(\infty))}$, который можно аппроксимировать как $S(\infty) \approx e^{-R_0}$ до тех пор, пока $R_0 \geq 2.5$.\\

\noindent Реализуем описанную выше модель для численности населения равной $N = 1000$. Проведем эксперименты, описанные в статье [1]. Сперва положим параметры равными $R_0=2$, $\frac{1}{\gamma}=14$, затем $R_0 = 3$, $\frac{1}{\gamma}=14$.

\begin{figure}[H]
    \center{\includegraphics[scale=0.6]{SIR_1.png}}
    \caption{Траектория модели SIR (1)}
	\label{fig:image}
\end{figure}

\begin{figure}[H]
    \center{\includegraphics[scale=0.6]{SIR_2.png}}
    \caption{Траектория модели SIR (2)}
	\label{fig:image}
\end{figure}

\noindent Видим, что полученные результаты полностью совпадают с требуемыми значениями.\\

\noindent Знаменитая идея модели SIR — это огромная ценность вакцинации, которая перемещает людей из группы $S$ непосредственно в группу $R$. Поскольку при $S < \frac{1}{R_0}$ распространение самоограничивается, вакцинация по крайней мере части населения $p_c=1 - \frac{1}{R_0}$ предотвратит вспышку. Это порог <<коллективного иммунитета>>. Практически в популяции все еще будут находиться восприимчивые люди, но патоген вызовет лишь короткую цепочку передачи с заиканием, потому что инфекционные люди вряд ли встретят достаточно восприимчивых.\\

\noindentОднако, SIR-модель перестает работать, если необходимо учитывать больше данных. Например, различную плотность населения в разных районах или разные пути передачи инфекции. Из-за очевидных недостатков SIR многократно дорабатывалась. Сегодня существует целое семейство моделей, разработанных на базе SIR-моделей, о которых пойдёт речь в дальнейшем.

\subsubsection{Модель SEIRS}

\noindent Простая, но фундаментальная структура SIR, представленная выше, была использована для получения важных сведений об эволюции новой эпидемии в идеализированной восприимчивой популяции со случайным смешиванием. Добавим в модель более сложные сценарии передачи болезней, введя новые группы и потоки между ними. Это позволит учесть такие аспекты как рождение, смерть, потеря иммунитета и возраст.\\

\noindent Для большинства инфекционных заболеваний существует скрытый период между заражением: группа, подвергшаяся контактированию с инфекцией ($E$). После заражения люди будут переходить в эту группу со скоростью $\frac{\beta S I}{N}$ и оставаться там в течение среднего периода $\frac{1}{\sigma}$, прежде чем перейти в группу $I$. При многих респираторных инфекциях иммунитет после выздоровления является временным, и выздоровевшие люди теряют иммунитет и возвращаются к $S$ после среднего защищенного периода $\frac{1}{\omega}$. \\

\noindent Демография способствует притоку в группы и выходу из них. Смерть из-за инфекции вызовет потерю людей из группы $I$ с коэффициентом $\alpha$, и все группы испытают фоновую смерть от других причин с коэффициентом $\mu$. В других стабильных популяциях фоновая смертность уравновешивается рождением в $S$ со скоростью $\mu N$.\\

\noindent Потеря иммунитета, рождение и смерть способствуют уязвимости в S, что создает <<открытую эпидемию>>. Эти размышления без учета вакцинации приводят нас к модели SEIRS:

\begin{center}
    \begin{cases}
      \frac{dS}{dt} = {\mu N} - {\frac{\beta I S}{N}} + {\omega R} - {\mu S} \\
       \frac{dE}{dt} = {\frac{\beta I S}{N}} - {\sigma E} -{\mu E}\\
       \frac{dI}{dt} = {\sigma E} - {\gamma I} - {(\mu + \alpha) I}\\
       \frac{dR}{dt} = {\gamma I} - {\omega R} - {\mu R}
    \end{cases}
\end{center}
\\

\noindent Соответсвующее базовое репродуктивное число равно $R_0 = [\frac{\sigma}{\sigma + \mu}] \times [\frac{\beta}{\gamma + \mu + \alpha}]$, потому что инфекционный период равен $\frac{1}{\gamma + \mu + \alpha}$ и вероятность того, что индексный случай станет заразным, а не умрет в период $E$, равняется $\frac{\sigma}{\sigma + \mu}$. Для большинства острых инфекций $\mu$ намного меньше, чем скорость эпидемии, поэтому реалистичные значения не меняют существенно траектории.\\

\noindent При реализации данной модели получили следующие графические представления траектории распространения, которые полностью удовлетворяют нашим ожиданиям. В параметрах модели изменяли значение $\beta$, которое в свою очередь влияло на коэффициент $R_0$:

\begin{figure}[H]
    \center{\includegraphics[scale=0.6]{SEIRS_2.png}}
    \caption{Траектория модели SEIRS (1)}
	\label{fig:image}
\end{figure}

\begin{figure}[H]
    \center{\includegraphics[scale=0.6]{SEIRS_1.png}}
    \caption{Траектория модели SEIRS (2)}
	\label{fig:image}
\end{figure}

\subsubsection{Модель SEIIR}
 
\noindent В данной интерпретации группа $I$ разделяется на две независимые между собой: $I_s,~ I_a$. \\

\noindent $I_s$ -- инфецированные с симптомами (Symptomatic Infectious). Люди из этой группы могут заражать восприимчивых, проявляют выраженные симптомы болезни.\\

\noindent $I_a$ -- инфецированные без выраженных симптомов (Asymptomatic Infectious). Люди могут заражать восприимчивых, однако видимых симптомов заболевания не наблюдается.\\

\noindent Примем, что $v_s$ -- процент симптоматично больных от общего числа инфецированных. Тогда $v_a = 1 - v_s$ -- процент бессимптомных больных. Таким образом, мы ввели вероятности перейти из группы $E$ в $I_s$ или $I_a$. Считаем, что средний латентный и инфекционный периоды у обеих групп инфекционных больных одинаковый.
$$A = \frac{1}{\alpha},~~~C=\frac{1}{\gamma}$$

\noindentАналогично $I$, разделим группу $R$ на тех, кто переболел и выздоровел бессимптомно ($R_a$) и тех, кто переболел, но были выражены признаки болезни ($R_s$).
Произведем нормировку: $S+E+I_s+I_a+R_s+R_a=1$. \\

\noindent Эти размышления приводят нас к системе дифференциальных уравнений следующего вида:

\begin{center}
    \begin{cases}
      \frac{dS}{dt} = -\beta (I_a + I_s)S\\
      \frac{dE}{dt} = \beta (I_a+I_s)S - \alpha E\\
      \frac{dI_s}{dt} = \alpha E v_s - \gamma I_s\\
      \frac{dI_a}{dt} = \alpha E v_a - \gamma I_a\\
      \frac{dR_s}{dt} = \gamma I_s\\
      \frac{dR_a}{dt} = \gamma I_a\\
    \end{cases}
\end{center}

\subsubsection{Модель SEIIRS}

\noindent Здесь вновь (как и в модели SEIRS) полагаем, что иммунитет приобретается человеком не на всегда. Рано или поздно человек из группы $R$ перейдёт в группу $S$. Введём вспомогательный коэффициент потери иммунитета к болезни (скорость перехода из группы $R$ в группу $S$) равный $\xi = \frac{1}{K}$, где $K$ -- длительность временного иммунитета.\\

\noindent Модель записывается в виде дифференциальных уравнений следущим образом:

\begin{center}
    \begin{cases}
      \frac{dS}{dt} = -\beta (I_a + I_s)S + \xi (R_a + R_s)\\
      \frac{dE}{dt} = \beta (I_a+I_s)S - \alpha E\\
      \frac{dI_s}{dt} = \alpha E v_s - \gamma I_s\\
      \frac{dI_a}{dt} = \alpha E v_a - \gamma I_a\\
      \frac{dR_s}{dt} = \gamma I_s - \xi R_s\\
      \frac{dR_a}{dt} = \gamma I_a - \xi R_a\\
    \end{cases}
\end{center}


\subsubsection{Изменяющаяся во времени скорость распространения инфекции}

\noindent Очень важным фактором является учёт вводимых карантинных мер в хоже протекания пандемии, что характеризуется изменением во времени значения параметра $\beta$, который ранее задавали как некоторую константу.\\

\noindent Введём в рассмотрение функцию:
$$f(t) = f_0 - \frac{1}{2}\left(1 + \tanh{\frac{t-t_1}{T}}\right)\left(f_0-f_1\right)$$

\noindent она имеет два фиксированных значения $f_0$ и $f_1$ и плавно изменяется от первого ко второму за промежуток времени $T$ с некоторой задержкой $t_1$. Аналогично изложенному принципу введём функцию $\beta(t)$. Будем придерживаться идеи, что было как минимум три фиксированных значения на протяжении наблюдаемого периода:

\begin{enumerate}
    \item $\beta_0$ -- значение параметра $\beta$ до начала пандемии, предполагается, что оно довольно велико.
    \item $\beta_1$ -- значение параметра $\beta$ в период самоизоляции.
    \item $\beta_2$ -- значение параметра $\beta$ после снятия жёстких карантинных мер.
\end{enumerate}

\noindent В дальнейшем в работе рассмотрим модель SEIIRS с $\beta(t)$, которая имела 4 значения. Каждое из слагаемых описывает плавное изменение значения функции $\beta(t)$ от значения $\beta_i$ к значению $\beta_i_+_1$, соответственно.\\

\noindent Введём функцию $\beta_3(t)$:
$$\beta_3(t) = \beta_0 - \frac{1}{2} \left(1 + \tanh{\frac{t-t_1}{T_1}} \right)(\beta_0-\beta_1) - \frac{1}{2} \left(1 + \tanh{\frac{t-t_1-b}{T_2}} \right)(\beta_1-\beta_2)$$

\noindent где $t_1$ -- параметр, характеризующий задержку реакции на принимаемые меры; $b$ -- примерное время начала перехода $\beta$ от $\beta_1$ к $\beta_2$ (начало второй волны); $\beta_1$ -- примерное время начало перехода $\beta$ от $\beta_2$ к $\beta_3$ (окончание второй волны); $T_1$ -- параметр, характеризующий скорость перехода $\beta_0 \rightarrow \beta_1$; $T_2$ -- параметр, характеризующий скорость перехода $\beta_1 \rightarrow \beta_2$.\\

\noindent Примерное время начала перехода $\beta$ от $\beta_0$ к $\beta_1$ принимается за начальный момент времени, поэтому в формуле не используется.\\

\noindent С введением $\beta(t)$, мы также вводим и новую важную характеристику $R_t(t)=\frac{\beta(t)}{\gamma}$ –- эффективное репродуктивное число, характеризующее среднее число вторичных случаев в популяции в определенный момент времени. Эта характеристика отличается от введенной ранее $R_0$ тем, что учитывает иммунитет в популяции и противоэпидемические меры.

\subsection{Обработка входных данных}

\noindent В качестве источника пользуемся сервисом Yandex DataLens, в котором публикуется статистика COVID-19 по регионам России. Данные обрабатывались с помощью библиотеки Pandas в среде Jupyter Notebook.\\

\noindent При загрузке данных, имеем исходную таблицу: 

\begin{figure}[H]
    \center{\includegraphics[scale=0.78]{table.JPG}}
	\label{fig:image}
\end{figure}

\noindent Создадим новую на основе имеющейся. Выберем из ней нужные столбцы, переименуем их. Далее выберем строки, которые относятся только к республике Башкортостан, произведя фильтрацию. Отсортируем их в хронологическом порядке. После этого будем иметь: 

\begin{figure}[H]
    \center{\includegraphics[scale=0.78]{bashkortostan.JPG}}
	\label{fig:image}
\end{figure}
 
\noindent Аналогичный образом, составляем сводные таблицы для других выбранных четырёх регионов.\\

\noindent Приведём графики зафиксированных случаев распространения COVID-19 в Башкортостане, Татарстане, Краснодарском крае, Крыму, ХМАО-Югре.

\begin{figure}[H]
    \center{\includegraphics[scale=0.59]{new_cases_bashk.jpg}}
    \caption{Случаи заражения в республике Башкортостан}
	\label{fig:image}
\end{figure}

\begin{figure}[H]
    \center{\includegraphics[scale=0.59]{new_cases_tatarstan.jpg}}
    \caption{Случаи заражения в республике Татарстан}
	\label{fig:image}
\end{figure}

\begin{figure}[H]
    \center{\includegraphics[scale=0.59]{new_cases_krasnodar.jpg}}
    \caption{Случаи заражения в Краснодарском крае}
	\label{fig:image}
\end{figure}

\begin{figure}[H]
    \center{\includegraphics[scale=0.59]{new_cases_krim.jpg}}
    \caption{Случаи заражения в Крыму}
	\label{fig:image}
\end{figure}

\begin{figure}[H]
    \center{\includegraphics[scale=0.59]{new_cases_hmao.jpg}}
    \caption{Случаи заражения в ХМАО-Югре}
	\label{fig:image}
\end{figure}

\noindent На всех графикам мы можем четко наблюдать первую и вторую волну эпидемии.

\subsection{Обучение модели}
\noindent Модель SEIIR с $\beta_3(t)$ обучалась на каждом из регионов и далее сравнивались распределения ошибок, получающиеся в результате многих запусков численной оптимизации параметров. При оптимизации параметров моделей минимизировалась сумма квадратов разностей между модельным общим числом зафиксированных заражений и реальными данными на каждый день.\\

\noindent Под общим числом зафиксированных заражений в терминах модели подразумевается общее число зафиксированных больных, проявлявших видимые симптомы болезни. Вычислить это значение можно, просуммировав число симптоматично инфекционных и переболевших симптоматично индивидуумов:
$$D_{model}= I_s+R_s$$

\noindent Минимизируемый функционал принимает вид:

$$\sum_{t=0}^{T}\left(D_{model}[t]-D_{real}[t]\right)^2$$

\noindent где $D_{real}[t]$ -- информация о новых случаях заражений в регионе в день $t$, $T$ -- длительность периода наблюдения (в днях).

\subsection{Численные эксперименты}
\subsubsection{Результаты для республики Башкортостан}

\noindent На следующем рисунке приведём боксплот распределения ошибок на обучающей выборке:

\begin{figure}[H]
    \center{\includegraphics[scale=0.59]{boxplot_bash.jpg}}
    \caption{Боксплот ошибок модели SEIIR по Башкортостану}
	\label{fig:image}
\end{figure}

\noindent Видим, что модель демонстрирует очень маленькие ошибки. Вычислим информационный критерий Акайке. Для используемой метрики (RSS) критерий вычисляется по формуле:
$$AIC = 2k + n ln (RSS_{min}) - (n ln(n) + 2C)$$
где $k$ -- количество параметров модели, $n$ -- размер выборки(число наблюдений), $C$ -- константа, зависящая от данных, $RSS_{min}$ -- минимальная ошибка модели, сумма квадратов разностей. 

\subsubsection{Результаты для республики Татарстан}

\subsubsection{Результаты для Краснодарского края}

\subsubsection{Результаты для Крыма}

\subsubsection{Результаты для ХМАО-Югра}

\section{Выводы}

\section{Заключение}

\begin{thebibliography}{9}

\bibitem{book1} Bjørnstad, O. N., Shea, K., Krzywinski, M. & Altman, N. Modeling infectious epidemics. NatureMethods17, 455–456 (2020)

\bibitem{book2} Bjørnstad, O.et al. The SEIRS model for infectious disease dynamics.NatureMethods17, 557–558(2020)

\bibitem{book3} Данные по динамике COVID-19 [Электронные ресурсы]:\\ https://github.com/beoutbreakprepared/nCoV2019/tree/master/latest\_data\\
https://www.healthmap.org/covid-19\\
https://www.worldometers.info/coronavirus\\

\end{thebibliography}

\end{document}
